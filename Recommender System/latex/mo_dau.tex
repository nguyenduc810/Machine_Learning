\chapter{MỞ ĐẦU}
\section{Lí do chọn đề tài}
Trong những năm gần đây, Trí tuệ nhân tạo (AI) đang thực sự bùng nổ với những bước tiến lớn mang đến phát minh mang tính ứng dụng cao. Cụ thể hơn là lĩnh vực học máy - Machine Learning với nhiều thuật toán mới giúp chúng ta tạo ra nhiều máy móc có tính tự động tốt. Nhờ những thành công của khoa học máy tính, sự xuất hiện của GPU, dẫn đến những cải tiến đáng kể trong hiệu suất của máy tính và sự phát triển của phần mềm đặc biệt, cho phép làm việc với dữ liệu lớn.\\
Trong khung chương trình đào tạo tại viện Toán ứng dụng và tin học, em được tiếp xúc với nhiều thuật toán, đó là lí do em muốn áp dụng những hiểu biết toán học của mình vào việc nghiên cứu các mô hình học máy.
Và em quyết định lựa chọn đề tài: "Machine Learning và ứng dụng trong hệ thống gợi ý". 
\section{Đối tượng nghiên cứu}
Đối tượng nghiên cứu trong đề tài là mô hình học máy - Machine Learning và ứng dụng của học máy trong hệ thống gợi ý.
\section{Phạm vi nghiên cứu}
Phạm vi nghiên cứu của đề tài là các thuật toán ứng dụng trong lĩnh vực AI - Machine learning.
\section{Ý nghĩa khoa học và thực tiễn}
\subsection{Ý nghĩa khoa học}
Kết quả nghiên cứu đồ án mang ý nghĩa khoa học rõ rệt, cho thấy:
\begin{itemize}
    \item Phương pháp phân tích ma trận giúp tìm ra các nhân tố ẩn tương quan trong ma trận.
    \item Sự trợ giúp của họ các phương pháp phân tích nhân tử ma trận giúp cho hệ thống gợi ý đạt được kết quả tốt.
\end{itemize}
\subsection{Ý nghĩa thực tiễn}
Kết quả của sự nghiên cứu có thể được ứng dụng để xây dựng hệ thống gợi ý các sản phẩm cho khách hàng được tốt hơn, đáp ứng kịp thời nhu cầu người dùng. Kết quả dự đoán giúp các doanh nghiệp tiếp cận được với đối tượng khách hàng phù hợp với sản phẩm của họ.
\newpage